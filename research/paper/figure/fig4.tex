\begin{figure}
  \setlength{\unitlength}{\textwidth}
\fbox{
  \begin{picture}(1,0.25)
    % % %90
    \put(0.02,0.03){\includegraphics[width=0.3\unitlength]{../FnP/gnuplot/fsi_displacement.eps}}
    \put(0.36,0.03){\includegraphics[width=0.3\unitlength]{../FnP/gnuplot/fsi_velocity.eps}}
    \put(0.72,0.03){\includegraphics[width=0.3\unitlength]{../FnP/gnuplot/fsi_power.eps}}
    
    \put(0.00,0.17){$\displaystyle{\frac{A}{D}}$}
    \put(0.33,0.17){$\displaystyle{\frac{V}{D}}$}
    \put(0.65,0.17){$\displaystyle{\frac{P_{m}}{\rho \mathcal{A}U^3 }}$}
    
    \put(0.18,0.0){\ustar} 	
    \put(0.51,0.0){\ustar}
    \put(0.87,0.0){\ustar}

    \put(0.08,0.22){\small(a)}
    \put(0.41,0.22){\small(b)}
    \put(0.77,0.22){\small(c)}

  \end{picture}  
}	
  \caption{Comparison of QSS (\ding{83}) and FSI (\ding{108}) data of displacement amplitude, velocity amplitude and mean power as a function of \ustar represented by (a), (b) and (c)respectively. Data were obtained at Re=165 and $\zeta=0.075$. An average error of $34\%$ could be observed for both displacement and velocity amplitude. Essential physics i.e the rise and fall of mean power could be captured of the FSI data}
    \label{fig:FSI_QSS_compare}
\end{figure}