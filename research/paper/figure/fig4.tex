\begin{figure}
  \setlength{\unitlength}{\textwidth}

  \begin{picture}(1,1.13)(0,0)
    
    % % %90
      \put(0.25,0.78){\includegraphics[width=0.5\unitlength]{../FnP/gnuplot/fsi_displacement.eps}}
      \put(0.25,0.4){\includegraphics[width=0.5\unitlength]{../FnP/gnuplot/fsi_velocity.eps}}
      \put(0.25,0.025){\includegraphics[width=0.5\unitlength]{../FnP/gnuplot/fsi_power.eps}}
     
   
	
            
      
      
   
 	\put(0.25,0.98){ \large $\frac{A}{D}$} 	
 	\put(0.25,0.62){\large $\frac{V}{D}$}
 	\put(0.2,0.23){\large $\frac{P_{m}}{\rho \mathcal{A}U^3 }$}
 	
% 	 	\put(0.25,0.88){ \ustar} 	
% 	 	\put(0.8,0.88){ \ustar}
 	 	\put(0.5,0.0){ \ustar}



    \put(0.36,1.1){(a)}
    \put(0.34,0.72){(b)}
    \put(0.34,0.34){(c)}
   
       

  \end{picture}  
	
  \caption{Comparison of QSS (\ding{83}) and FSI (\ding{108}) data of displacement amplitude, velocity amplitude and mean power as a function of \ustar represented by (a), (b) and (c)respectively. Data were obtained at Re=165 and $\zeta=0.075$. An average error of $34\%$ could be observed for both displacement and velocity amplitude. Essential physics i.e the rise and fall of mean power could be captured of the FSI data}
    \label{fig:FSI_QSS_compare}
\end{figure}